Technology games. As Coeckelbergh pointed out, it is the task of technology games to reveal the hidden depth grammars of technology. The scope of properly addressing OKM in terms of technology games and its
grammar is far too extensive for this brief section. A social,
historical, and cultural account of academic search engines might be an
interesting task for the future, but for now I hope to show an way of
thinking about OKM with technology games in mind. Table 1 gave a (very) elementary and subjective overview of a few communication channels and their characteristics. To the avid user of the internet it comes as no surprise that the directionality of communication has changed in the recent decades with the conception of the internet and the recent rise of the web 2.0\footnote{Web 2.0 is a term that usually refers to the new generation of interactive, user-focussed internet applications such as social media, wikis, or collaborative platforms.} Nevertheless, scholarly publishing has not adopted web 2.0 technologies thoroughly, yet. Open Science, open access and similar initiatives are advocating for a change in scholarship, but the discovery process of knowledge is still dominated by the traditional approaches that were conceived in the context of archiving and library sciences. The idea of browsing lists of individual entries based on keywords was useful when the review of texts on a bigger scale was simply not feasible, but modern technology and the increased acceptance of Open Access are game changers. OKM and its design is strongly grounded in this cultural and historical changes of scholarship. Thus, the study of these changes is not only relevant to the historians and sociologists of science, but being aware of the existing grammars of technology could be essential to a technology of scholarly communication in the 21st century.