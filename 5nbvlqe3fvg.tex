Coeckelbergh (2017, p.15) identifies three tasks for a “Wittgensteinian holistic, transcendental, and critical phenomenology and hermeneutics of
technology”: (1) the revelation of both surface and depth grammar of technology, (2) reveal the normativity of those technology
grammars\footnote{We can not use technology without following the rules imposed on us by the social and historical context}, and (3) make us
aware of the active, “game-changing”\footnote{A great example for a game-changer might be the introduction of social media. Nobody
would question the manifold ways that this technology influenced the very nature of communication, socialising, and even politics.} nature
of technology and forsake the idea of the neutral tool. Coeckelbergh
concludes that “thinking about technology is also thinking about the
ways we do things, and ultimately about our world and an entire form of
life.” It is interesting to note that while this way of thinking might
be novel in the context of the philosophy of technology, engineers and
designers had adopted a similar way of thinking and conceiving
technology many years ago. User-centric design, soft ergonomics, or
human-computer interaction are a few examples of technology design
strategies that were immensely popularised by works such as User
Centered System Design: New perspectives on human-computer interaction
(Donald and Draper, 1986) or the more recent Designing with the
mind in mind: simple guide to understanding user interface design
guidelines (Johnson, 2010). Awareness for the cultural, social, and
historical embedding of a user of technology have been part of software
development and engineering for some time and an extensive investigation
of interface design strategies might be an interesting future task in
the context of technology games .