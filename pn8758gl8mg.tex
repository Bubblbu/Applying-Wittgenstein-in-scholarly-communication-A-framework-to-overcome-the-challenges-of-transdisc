OKM is a visual knowledge discovery tool that provides an overview of a research field based on available metadata and abstracts of relevant articles. The full technical implementation and code is available online as Open Source code\footnote{https://github.com/OpenKnowledgeMaps/Headstart}. The authors furthermore have described the future work and features that are planned (Kraker, Kittel and Enkhbayar, 2016). The basic search functionality resembles any other standard academic search engine, but the results are then presented in an interactive, what the authors call,
knowledge map . These knowledge maps consist of individual items
(publications) and topic bubbles which are calculated based on the
available metadata abstracts of the items (see fig. 1) While the team’s
current focus is mostly on the improvement of the current language
processing algorithms in order to increase the quality of the extracted
topics, groupings, and labelling, I want to selectively zone in on a few
aspects that might be of more interest than the algorithmic
implementation. Specifically I want to, once again, return to the
conceptual metaphor theory and language games to see how
OKM might be addressing certain issues in scholarly communication
differently than other solutions.