While one could argue that the bulk of Wittgenstein’s work deals with
(the mysteries) of communication, in this essay I want to address two
new domains that Wittgenstein did not address in the specific way I am.
The first one is scholarly communication , which is quite a
special case of our language as I believe that the academic setting and
highly specialised terminology perfectly fits the critical use-cases of
language that Wittgenstein has been talking about. The second domain is
technology . Times have changed since Wittgenstein’s musings about
the nature of philosophy, and are still drastically changing with
ever-accelerating technological advances. Unsurprisingly the philosophy
of technology has been established as a discipline of its own. This is
also why the journal Techné: Research in Philosophy and
Technology has issued a special call about “Wittgenstein and the
philosophy of technology” and why I believe that this investigation
into Wittgenstein and scholarly communication in the light of modern
technology qualifies as a domain that is worth to venture into. Just as
Wittgenstein calls for awareness of the situatedness of the words in our
everyday language \citep(Wittgenstein, 1953, §116), I want to situate the
theories of scholarly communication in the everyday reality of scholarly
communication. While many researchers of scholarly communication are
addressing questions such as “why” and “how” researchers communicate
in today’s society, I want to propose a technical solution qualifies as
a “resolute practice of Wittgensteinian philosophy” (Read, 2007,
p.134). To do so I want to first discuss the nature of transdisciplinary
communication and the challenges that arise with it. I will give an
anecdotal overview of what modern scholarly communication entails and
how advancing technology is transforming it. Having established a basic
terminology and outlined the problems related to scholarly
communication, I will briefly review what it means to “apply
Wittgenstein” while keeping his view of “philosophy as therapy” in
mind. Next, I will address the kinds of confusions caused by
transdisciplinary communication and technology respectively. I want to
discuss the central role that metaphors play in Wittgenstein’s work and
then introduce Lakoff & Johnson’s (2003) conceptual metaphor
theory as an applied framework to analyse and think about metaphors in
our cognition. Mark Coeckelbergh’s (2017) language games will
serve as a framework to understand technology in Wittgensteinian terms
and will help to understand social, historical, and cultural challenges
in technology. Finally, I will briefly analyse Open Knowledge Maps, a
visual knowledge discovery tool, in terms of the conceptual
metaphor theory and technology games in order to show how this
approach is partially being applied in a real life setting. I am aware
this attempt to cover various levels of analysis across multiple
disciplines and theories and even trying to gap the bridge between
theory and application, might fail to adequately draw the bigger
picture, but I strongly believe that any attempt to understand and apply
Wittgenstein requires “taking in a domain of thought or life that is of
some moment beyond the academy ” (Read, 2007, p.134).
Understanding the risks of widening my focus, I hope to, nevertheless,
show how this line of thought could not only prove to be fruitful to
scholarly communication, but Wittgenstein scholars and philosophers of
technology as well.
